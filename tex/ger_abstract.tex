
Lebende Organismen m\"ussen st\"andig Entscheidungen treffen, die sich auf partieller, ungewisser Information basieren. Nicht nur des Organismus Umfeld ist unsicher, sondern auch die
komputationellen Einheiten, die es zur Bildung einer Entscheidung anwendet, wie zum Beispiel, Neuronen in Nervensystemen. Immerhin besteht zwischen den beiden ein wichtiger
Unterschied: auf die Umwelt kann das Organismus nur indirekt agieren, w\"ahrend die Neuronen direkt adaptiv eingestellt werden k\"onnen. Ich befasse mich hier mit der Frage, wie
Populationen von Neuronen sich organisieren k\"onnen um die Informationsverarbeitung seiner Ausg\"ange zu erleichtern. Darin fokussiere Ich mich auf dynamischen Stimuli, die
schnelle Entscheidungen erbitten.\par

In diesem Umfeld habe Ich mehrerer neue Ergebnisse erreicht, und das Feld der Populationskodierung zu einem dynamischen Kontext ausgearbeitet, in dem eine dynamische Umwelt
\emph{online} dekodiert wird, also in Echtzeit. Dazu habe Ich die Theorie der Filterung doppelstochastischer Punktprozesse angewendet. F\"ur dichte Populationen von Neuronen kann
das als ein Gaussprozessregressionsproblem formuliert werden, f\"ur welchen Ich ein geschlossenen Ausdruck f\"ur die Verteilung der Fehler fand. Das erarbeitete Formalismus erlaubt eine direkte
Erweiterung zu einem kontrolltheoretischen Kontext, und Ich habe die Ergebnisse in der kontrolltheoretischen Formulierung mit den Ergebnissen in der stochastischen Filterung verglichen.
Interessanterweise, war es mir m\"oglich zu zeigen, dass beide Aufgaben zu unterschiedlichen optimalen Kodierungsstrategien f\"uhren.\par

Das Feld der optimalen Kodierung in neuronalen Populationen ist ein aktives Forschungsfeld. Ich habe hier mehrere Ergebnisse zum Fall von dynamischen Stimuli erweitert. Parallel
habe Ich diese Ergebnisse mit Kontrol-theoretischen Ergebnissen verglichen und diskutiert welche Implikationen das hat.