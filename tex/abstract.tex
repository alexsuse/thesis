
Organisms are routinely faced with the task of making decisions based on partial, unreliable information. Furthermore, not only is the environment noisy, but the computational units 
organisms employ themselves are often faulty and noisy, such as the neurons of the animal nervous system. There is a crucial difference, though: while the organism can only act
indirectly on its environment, its sensory organs are under its direct control through adaptation. Here I thus consider how a population of neurons can organise itself to allow for optimal information 
processing at its output. I will focus on a model of a rapidly changing environment, which forces the organism to make short-time decisions.\par

In that setting I have provided a number of novel results, extending the framework of population coding to a fully dynamic setting, where a changing environment is decoded from
spike trains in an online fashion. To do so, I have drawn from the theory of stochastic filtering of doubly stochastic point processes. For a dense population of neurons, this
reduces to a Gaussian process regression problem, for which I have obtained closed form relations for the error distribution. The formalism developed also lends itself to a direct 
extension to a control-theoretical setting. Thus I have also developed a criterion for optimal coding
in the case of control problems, and have compared this to the case of estimation previously developed. Interestingly, I have been able to show that the different objective functions
for control and estimation lead to different optimal encoders.\par

The study of optimal coding in populations of neurons is an active fertile area of research. Here I have extended a number of previous findings to the case of online decoding of
dynamic stimuli from point processes. Parallel to that, I have discussed the issue of optimal coding for control, and how it relates to the study of optimal coding in the estimation case.



