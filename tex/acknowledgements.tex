
In the media and among friends, it seems a particular narrative of personal success is becoming more and more popular. I mean the narrative of \emph{having made it on my own}, or 
\emph{building my own success} and \emph{being responsible for my own success}. I personally find this kind of notion arrogant if not ludicrous, and operate under no such assumption.
In the timeless words of Billy Shears, \emph{I get by with a little help from my friends}. Let me then name a few of them.\par

None of this, my coming to Berlin, my doctoral studies, would have happened if I had not met Manfred Opper, and I have come to consider him a friend as well as a mentor. 
Our discussions have always touched on fascinating subjects, such as the tale of an austrian nobleman who came to be executed as the emperor of Mexico, the story of the siege of 
Jerusalem and classic rock facts, occasionally mentioning functional analysis, function-valued stochastic processes and information theory. I also wish to thank the KI group he leads, 
Florian, Cordula, Andreas, Philipp and more recently Ludovica, for providing a supportive, relaxed and fun work environment with a lot of stimulating discussions.\par

There are many friends I have made in Berlin, and though it is hard to single one out, Chris is surely the one who shared most of my experience here. In the meantime we have
shared a research project, become friends and become parents, and it has been a great joy to share these experiences with him, with his awesome wife Katrin and his even more 
awesome daughter Florence. Thank you very much guys, you are missed.\par

One of the greatest things about my time in Berlin was the opportunity to meet so many different people, and learning how much we share. Sinem has
taught me how similar brazilian and turkish people are, and has been a great research partner and friend. Our lunchtimes will be missed, Sinem. Though it was hardly a surprise,
Tommaso also taught me how much Italy has in common with Brazil, and the warmth of your friendship has made me feel at home away from home. There were so many others it
is hard to name them all, but I can not refrain from mentioning Joachim, always fun company with cool board games and weird factoids, Fred, the BCCN's smiling DJ, always uplifting
and bringing in some seriously cool disco songs, and Philippe, our awesome canadian who can easily go five hours on a single conversation and will not let the night end a second
too early.\par

Someone once told me: \emph{Friends come and go, enemies accumulate.} I must be a very lucky person then, since, unless I am completely oblivious to it, I have managed to avoid
making blood feuds while holding on to some great friends from older times. Marcos and Flavia, whom we followed to Berlin, have always been a source of support, and we have not 
met nearly as much as I'd like to in these last years. Domingos, who followed us to Berlin to live the student life, remains a great friend, and it was a great pleasure being able to
share some of my Berlin time with you. And finally, Igor, who from afar has still kept us close to his daily life. Thank you all.\par

Naturally, a lot of my time in Berlin has actually been spent with academic work, and I would like to thank my collaborators. Ronny has always been a great collaborator and advisor, and 
I sincerely
do not believe my papers would have made as much sense if he hadn't helped set the context and organise our ideas into a coherent body of research. I thank him for the collaboration
and for the patience. Martin, Chris' supervisor, has also been a great advisor, and has helped us put a research project on firm standing and framing it in a neuroscientific context. I am,
however, most in debt to Vanessa Casagrande for making my work possible. Besides helping in navigating the german bureaucracy, securing soft skill offerings, organising retreats 
and research visits, you have always been a source of support and guidance for me and surely for most of the students in the GRK. Thank you for that and for being a great friend.\par

Most of all, though, I am in debt for my family for supporting me and making this possible. First of all, I would like to thank my wife, Fernanda, for joining me in moving across the globe
and supporting me unconditionally. That is more than I could have hoped for, but you went overboard and presented me with the greatest gift I could imagine, our little Hugo, and
for that I am forever grateful to you. You two are my inspiration. My parents and my brother have always been there for me, and I am forever thankful for your support, your love
and the interest you take in whatever crazy project I decide to drag myself into. If I have ever felt safe to try my luck in the world, it was because I knew I had a safe haven to turn back to.
%
%
%It is hard to look back on this long a time and list the awesome people I have met along the way. Berlin has been a home for me for almost four years, and in those years more
%has happened than I could ever have foreseen when I moved here. 
%\par
%I would like to thank my wife, Fernanda, who made this time in Berlin full of joy and light, even
%in the darkest winter months. When I first moved to Berlin, I knew I had a lot of work ahead, but I believed most of it would be academic. A little guy came along and has shown me
%otherwise. It has been a great ride for these past 15 months with Hugo, and that is one more reason I need to thank my wife for.\par
%
%Working in academia one routinely hears stories about exploratory work environments, late hours, disrespectful superiors, but I am glad to say that my experience was exactly the
%opposite. Manfred has provided me with enough freedom to pursue my own interests and the necessary guidance towards the right research problems. I have yet to encounter a
%person who shows the same passion and enthusiasm for research as Manfred and he has hooked me on path integrals for life. I would like to thank Manfred and all my colleagues at 
%the Artificial Intelligence group for the relaxed and nurturing work environment they provided.\par
%
%The field of neuroscience is vast and I have been lucky to participate in a graduate school that has not only provided me with a great outlook onto the topic, but also gave me
%great friends. I would like to thank all of you who have made this time so fun and interesting. Specially I would like to thank Sinem, for the long lunches and great conversations on
%random topics, Tommaso, for the endless coffees and the great scientific discussions and Joachim, who was never short of interesting and curious conversations to have. I would
%also like to thank Vanessa, who navigated my way through the endless bureaucracies and was always helpful and available when needed.\par
%
%In my time in Berlin I was lucky enough to extend my family, and it was a great joy to see a good friend do so as well. Chris has been a great and fun collaborator, but he was foremost
%a friend, and 
%
%
%None of this would have happened if it weren't for my advisor, Manfred Opper, so I will start with him. I have learned
%
%First and foremost I'd like to thank my wife, Fernanda, for following through with my craziest plans. I could never had dreamed of finding such a good friend for life.